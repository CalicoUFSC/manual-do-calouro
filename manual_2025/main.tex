\documentclass{article}

\usepackage{fullpage}
\usepackage[brazilian]{babel}
\usepackage[utf8]{inputenc}
\usepackage[T1]{fontenc}

\usepackage[a5paper]{geometry}
\renewcommand{\familydefault}{\sfdefault}
\usepackage[scaled=1]{helvet}
\usepackage[helvet]{sfmath}
\everymath={\sf}
\usepackage{pdflscape}
\usepackage{rotating}

\usepackage{tikz}
\usetikzlibrary{calc}

\usepackage{graphicx}
\graphicspath{{./images/}}

\usepackage{parskip}
\usepackage[colorinlistoftodos]{todonotes}
\usepackage[colorlinks=true, allcolors=blue]{hyperref}

\title{Manual Do Calouro}
\author{CALICO - Centro Acadêmico Livre da Computação}
\setcounter{tocdepth}{2}

\begin{document}


% Capa
\newgeometry{margin=0em,top=0em}
    \vspace{5cm}
    \includegraphics[width=\paperwidth,height=\paperheight]{capa}
\restoregeometry
% /Capa

\newpage
\thispagestyle{empty} % No header/footer on the blank page
\mbox{} % Adds an empty box, effectively creating a blank page
\newpage

\newpage
\thispagestyle{empty} % No header/footer on the blank page
\mbox{} % Adds an empty box, effectively creating a blank page
\newpage

\maketitle
\tableofcontents

\clearpage


\section {Órgãos e Departamentos}

\subsection{RU}

Apesar das promessas de um novo RU e do crescente descaso e sucateamento do que já temos, continuamos com exatamente dois RU na UFSC de Floripa.

Um no campus Trindade, onde você possivelmente terá todas as aulas da sua graduação.
E um no campus do CCA (Centro de Ciências Agrárias), que fica num bairro um pouco mais distante chamado Itacorubi. A comida lá é muito mais temperada e frequentemente melhor que o do campus Trindade. Não deixe de provar quando tiver oportunidade.

O preço padrão do RU é R\$1,50. Pode ser gratuito em casos específicos como na comprovação de baixa renda (basta requisitar na PRAE).
O preço do RU também pode ser mais caro nos casos para professores, servidores e pessoas externas da universidade.
Sim, todos podem comer no RU, porém se a pessoa não tiver vínculo com a UFSC (ou tiver quebrado/esquecido o cartão do RU) precisa pagar o preço da refeição sem o subsídio da universidade(que ainda é muito mais barato que qualquer alternativa de refeição completa nos arredores).

Horário:
Segunda a sexta: 11h às 13:30h e das 17h às 19h

Sábados, domingos e feriados: 11h às 13:00h e das 17h às 19h

Cardápio:
O Cardápio do RU é disponibilizado semanalmente toda segunda-feira através do site \href{https://ru.ufsc.br/ru/}{ru.ufsc.br/ru/} e diariamente no instagram \href{https://www.instagram.com/ru360ufsc/}{@ru360ufsc}.

Também existe o amado \href{https://t.me/quibebot}{@quibebot} no Telegram, porém ele precisa de uma pequena manutenção a cada vez que a formatação do PDF do cardápio disponibilizado é modificado, geralmente a cada novo semestre. Aspirantes a heróis do curso, contatem o CALICO.

O cardápio é o mais provável, mas não garantido, não é incomum o cardápio mudar sem aviso prévio, mas normalmente, quando mudado, é para um prato similar ou Sassami.
A comida normalmente é composta por Arroz parboilizado e integral, feijão ou lentilha, um complemento variado (Farofa, batata palha, legume etc) uma porção de proteína, saladas e sobremesa. 
Dicas sobre o RU: 

Tome cuidado, o RU possui duas filas, uma na entrada frontal, que se desloca pela lateral do conviva, e outra que vai da porta dos fundos, pela lateral e saída do ru, até na frente da secretaria do RU, não misturem as duas, vai dar problema e chatice.
Se possível, priorize horários com menos movimento, como antes do 12:00 ou após 12:30.
Utilize todas as catracas disponíveis, não fique esperando alguém passar na catraca se tem uma disponível do lado.

\subsubsection{Carteirinha} Para conseguir acessar o RU, é necessário uma carteirinha e passes.
Na primeira semana costumava ser tolerado que os calouros apresentem seus atestados de matrícula para conseguirem entrar, já que as filas para fazer a carteirinha presencialmente eram muito grandes.
Atualmente a solicitação do cartão é apenas online e não é possível retirar o cartão no mesmo dia.
Então favor encher o saco da secretaria do RU para voltar o período de tolerância.

A solicitação é realizada no seguinte endereço: \href{https://ru.ufsc.br/solicitacao-online/}{ru.ufsc.br/solicitacao-online}.
Escolha uma foto boa, ela vai ficar na sua carteirinha para tudo, e se for recusada, vai atrasar o processo todo.

Todos os dados necessários para preencher a requisição podem ser encontrados no seu CAGR.

\subsubsection{Passes} 
Todos passes do cartão são recarregados apenas online: \href{https://ru.ufsc.br/pagamento-por-pix/}{ru.ufsc.br/ pagamento-por-pix}.
Enquanto que os passes físicos para pessoas externas são comprados na secretaria do RU (ao lado do RU) no caso do campus Trindade. No campus do CCA, esses passes são comprados na cantina.

\subsection{BU}
A Biblioteca Universitária (BU, lê-se “bê-u”) é o local de estudos mais usado pelos estudantes. Lá, você consegue total silêncio e tem a sua disposição um acervo enorme. É utilizada também para estudos em grupos. Se você quiser ainda mais concentração, a BU disponibiliza salas de estudos individuais, no piso térreo.

Para fazer um empréstimo, você deve ir até o balcão de atendimento com um documento de identificação oficial  e fazer seu cadastro. Com o cadastro feito, você pode consultar o acervo online, disponível no site bu.ufsc.br e localizar o seu livro. Depois, é só levar seu livro ao balcão e efetuar o empréstimo. Podem ser emprestados até cerca de 10 livros (varia de acordo com a seção), com o prazo de até duas semanas. Mas, você pode sempre renovar o seu empréstimo no próprio site da BU.

Horário de Funcionamento:


Segunda a sexta das 7:30 ás 22h

Sábados, Domingos e feriados: A sala de estudos individuais fica aberta das 8 às 17h

\subsection{LabUFSC}
No prédio da biblioteca, ao lado da entrada principal, fica o LabUFSC. Nele, há vários computadores com acesso a internet que podem ser usados gratuitamente pelos alunos. Para entrar, você só precisa apresentar sua carteirinha do RU na entrada.

Horário de Funcionamento:

Segunda a sexta das 8 às 22h

Sábados e Domingos das 8 às 17h

\subsection{PRAE} A Pró-Reitoria de Assuntos Estudantis – PRAE é um órgão voltado especialmente para programas e projetos ligados à política estudantil. É pra ela que se deve recorrer para saber sobre: Bolsas Estudantis, Auxílio Moradia, Moradia Estudantil, etc. Se você tem uma renda familiar bruta mensal de até 1,5 salário mínimo per capita (os que fizeram comprovação de renda já estão cadastrados), pode solicitar o seu Cadastro PRAE e então se inscrever para conseguir os benefícios. 

Os editais são lançados no site:

\url{http://prae.ufsc.br/editais-por-programa/}

\subsection{INE}
O Departamento de Informática e Estatística (INE) é o departamento responsável pelo nosso curso e o de Sistemas da Informação. Em seus cinco andares, possui cerca de 20 laboratórios (que serão apresentados) e é onde se encontram a secretaria e a coordenação do curso (térreo). Há também salas de professores e salas de aulas onde se realizam atividades práticas.


\section {Órgãos e Departamentos}

\subsection{RU}

Apesar das promessas de um novo RU e do crescente descaso e sucateamento do que já temos, continuamos com exatamente dois RU na UFSC de Floripa.

Um no campus Trindade, onde você possivelmente terá todas as aulas da sua graduação.
E um no campus do CCA (Centro de Ciências Agrárias), que fica num bairro um pouco mais distante chamado Itacorubi. A comida lá é muito mais temperada e frequentemente melhor que o do campus Trindade. Não deixe de provar quando tiver oportunidade.

O preço padrão do RU é R\$1,50. Pode ser gratuito em casos específicos como na comprovação de baixa renda (basta requisitar na PRAE).
O preço do RU também pode ser mais caro nos casos para professores, servidores e pessoas externas da universidade.
Sim, todos podem comer no RU, porém se a pessoa não tiver vínculo com a UFSC (ou tiver quebrado/esquecido o cartão do RU) precisa pagar o preço da refeição sem o subsídio da universidade(que ainda é muito mais barato que qualquer alternativa de refeição completa nos arredores).

Horário:
Segunda a sexta: 11h às 13:30h e das 17h às 19h

Sábados, domingos e feriados: 11h às 13:00h e das 17h às 19h

Cardápio:
O Cardápio do RU é disponibilizado semanalmente toda segunda-feira através do site \href{https://ru.ufsc.br/ru/}{ru.ufsc.br/ru/} e diariamente no instagram \href{https://www.instagram.com/ru360ufsc/}{@ru360ufsc}.

Também existe o amado \href{https://t.me/quibebot}{@quibebot} no Telegram, porém ele precisa de uma pequena manutenção a cada vez que a formatação do PDF do cardápio disponibilizado é modificado, geralmente a cada novo semestre. Aspirantes a heróis do curso, contatem o CALICO.

O cardápio é o mais provável, mas não garantido, não é incomum o cardápio mudar sem aviso prévio, mas normalmente, quando mudado, é para um prato similar ou Sassami.
A comida normalmente é composta por Arroz parboilizado e integral, feijão ou lentilha, um complemento variado (Farofa, batata palha, legume etc) uma porção de proteína, saladas e sobremesa. 
Dicas sobre o RU: 

Tome cuidado, o RU possui duas filas, uma na entrada frontal, que se desloca pela lateral do conviva, e outra que vai da porta dos fundos, pela lateral e saída do ru, até na frente da secretaria do RU, não misturem as duas, vai dar problema e chatice.
Se possível, priorize horários com menos movimento, como antes do 12:00 ou após 12:30.
Utilize todas as catracas disponíveis, não fique esperando alguém passar na catraca se tem uma disponível do lado.

\subsubsection{Carteirinha} Para conseguir acessar o RU, é necessário uma carteirinha e passes.
Na primeira semana costumava ser tolerado que os calouros apresentem seus atestados de matrícula para conseguirem entrar, já que as filas para fazer a carteirinha presencialmente eram muito grandes.
Atualmente a solicitação do cartão é apenas online e não é possível retirar o cartão no mesmo dia.
Então favor encher o saco da secretaria do RU para voltar o período de tolerância.

A solicitação é realizada no seguinte endereço: \href{https://ru.ufsc.br/solicitacao-online/}{ru.ufsc.br/solicitacao-online}.
Escolha uma foto boa, ela vai ficar na sua carteirinha para tudo, e se for recusada, vai atrasar o processo todo.

Todos os dados necessários para preencher a requisição podem ser encontrados no seu CAGR.

\subsubsection{Passes} 
Todos passes do cartão são recarregados apenas online: \href{https://ru.ufsc.br/pagamento-por-pix/}{ru.ufsc.br/ pagamento-por-pix}.
Enquanto que os passes físicos para pessoas externas são comprados na secretaria do RU (ao lado do RU) no caso do campus Trindade. No campus do CCA, esses passes são comprados na cantina.

\subsection{BU}
A Biblioteca Universitária (BU, lê-se “bê-u”) é o local de estudos mais usado pelos estudantes. Lá, você consegue total silêncio e tem a sua disposição um acervo enorme. É utilizada também para estudos em grupos. Se você quiser ainda mais concentração, a BU disponibiliza salas de estudos individuais, no piso térreo.

Para fazer um empréstimo, você deve ir até o balcão de atendimento com um documento de identificação oficial  e fazer seu cadastro. Com o cadastro feito, você pode consultar o acervo online, disponível no site bu.ufsc.br e localizar o seu livro. Depois, é só levar seu livro ao balcão e efetuar o empréstimo. Podem ser emprestados até cerca de 10 livros (varia de acordo com a seção), com o prazo de até duas semanas. Mas, você pode sempre renovar o seu empréstimo no próprio site da BU.

Horário de Funcionamento:


Segunda a sexta das 7:30 ás 22h

Sábados, Domingos e feriados: A sala de estudos individuais fica aberta das 8 às 17h

\subsection{LabUFSC}
No prédio da biblioteca, ao lado da entrada principal, fica o LabUFSC. Nele, há vários computadores com acesso a internet que podem ser usados gratuitamente pelos alunos. Para entrar, você só precisa apresentar sua carteirinha do RU na entrada.

Horário de Funcionamento:

Segunda a sexta das 8 às 22h

Sábados e Domingos das 8 às 17h

\subsection{PRAE} A Pró-Reitoria de Assuntos Estudantis – PRAE é um órgão voltado especialmente para programas e projetos ligados à política estudantil. É pra ela que se deve recorrer para saber sobre: Bolsas Estudantis, Auxílio Moradia, Moradia Estudantil, etc. Se você tem uma renda familiar bruta mensal de até 1,5 salário mínimo per capita (os que fizeram comprovação de renda já estão cadastrados), pode solicitar o seu Cadastro PRAE e então se inscrever para conseguir os benefícios. 

Os editais são lançados no site:

\url{http://prae.ufsc.br/editais-por-programa/}

\subsection{INE}
O Departamento de Informática e Estatística (INE) é o departamento responsável pelo nosso curso e o de Sistemas da Informação. Em seus cinco andares, possui cerca de 20 laboratórios (que serão apresentados) e é onde se encontram a secretaria e a coordenação do curso (térreo). Há também salas de professores e salas de aulas onde se realizam atividades práticas.


\section {Órgãos e Departamentos}

\subsection{RU}

Apesar das promessas de um novo RU e do crescente descaso e sucateamento do que já temos, continuamos com exatamente dois RU na UFSC de Floripa.

Um no campus Trindade, onde você possivelmente terá todas as aulas da sua graduação.
E um no campus do CCA (Centro de Ciências Agrárias), que fica num bairro um pouco mais distante chamado Itacorubi. A comida lá é muito mais temperada e frequentemente melhor que o do campus Trindade. Não deixe de provar quando tiver oportunidade.

O preço padrão do RU é R\$1,50. Pode ser gratuito em casos específicos como na comprovação de baixa renda (basta requisitar na PRAE).
O preço do RU também pode ser mais caro nos casos para professores, servidores e pessoas externas da universidade.
Sim, todos podem comer no RU, porém se a pessoa não tiver vínculo com a UFSC (ou tiver quebrado/esquecido o cartão do RU) precisa pagar o preço da refeição sem o subsídio da universidade(que ainda é muito mais barato que qualquer alternativa de refeição completa nos arredores).

Horário:
Segunda a sexta: 11h às 13:30h e das 17h às 19h

Sábados, domingos e feriados: 11h às 13:00h e das 17h às 19h

Cardápio:
O Cardápio do RU é disponibilizado semanalmente toda segunda-feira através do site \href{https://ru.ufsc.br/ru/}{ru.ufsc.br/ru/} e diariamente no instagram \href{https://www.instagram.com/ru360ufsc/}{@ru360ufsc}.

Também existe o amado \href{https://t.me/quibebot}{@quibebot} no Telegram, porém ele precisa de uma pequena manutenção a cada vez que a formatação do PDF do cardápio disponibilizado é modificado, geralmente a cada novo semestre. Aspirantes a heróis do curso, contatem o CALICO.

O cardápio é o mais provável, mas não garantido, não é incomum o cardápio mudar sem aviso prévio, mas normalmente, quando mudado, é para um prato similar ou Sassami.
A comida normalmente é composta por Arroz parboilizado e integral, feijão ou lentilha, um complemento variado (Farofa, batata palha, legume etc) uma porção de proteína, saladas e sobremesa. 
Dicas sobre o RU: 

Tome cuidado, o RU possui duas filas, uma na entrada frontal, que se desloca pela lateral do conviva, e outra que vai da porta dos fundos, pela lateral e saída do ru, até na frente da secretaria do RU, não misturem as duas, vai dar problema e chatice.
Se possível, priorize horários com menos movimento, como antes do 12:00 ou após 12:30.
Utilize todas as catracas disponíveis, não fique esperando alguém passar na catraca se tem uma disponível do lado.

\subsubsection{Carteirinha} Para conseguir acessar o RU, é necessário uma carteirinha e passes.
Na primeira semana costumava ser tolerado que os calouros apresentem seus atestados de matrícula para conseguirem entrar, já que as filas para fazer a carteirinha presencialmente eram muito grandes.
Atualmente a solicitação do cartão é apenas online e não é possível retirar o cartão no mesmo dia.
Então favor encher o saco da secretaria do RU para voltar o período de tolerância.

A solicitação é realizada no seguinte endereço: \href{https://ru.ufsc.br/solicitacao-online/}{ru.ufsc.br/solicitacao-online}.
Escolha uma foto boa, ela vai ficar na sua carteirinha para tudo, e se for recusada, vai atrasar o processo todo.

Todos os dados necessários para preencher a requisição podem ser encontrados no seu CAGR.

\subsubsection{Passes} 
Todos passes do cartão são recarregados apenas online: \href{https://ru.ufsc.br/pagamento-por-pix/}{ru.ufsc.br/ pagamento-por-pix}.
Enquanto que os passes físicos para pessoas externas são comprados na secretaria do RU (ao lado do RU) no caso do campus Trindade. No campus do CCA, esses passes são comprados na cantina.

\subsection{BU}
A Biblioteca Universitária (BU, lê-se “bê-u”) é o local de estudos mais usado pelos estudantes. Lá, você consegue total silêncio e tem a sua disposição um acervo enorme. É utilizada também para estudos em grupos. Se você quiser ainda mais concentração, a BU disponibiliza salas de estudos individuais, no piso térreo.

Para fazer um empréstimo, você deve ir até o balcão de atendimento com um documento de identificação oficial  e fazer seu cadastro. Com o cadastro feito, você pode consultar o acervo online, disponível no site bu.ufsc.br e localizar o seu livro. Depois, é só levar seu livro ao balcão e efetuar o empréstimo. Podem ser emprestados até cerca de 10 livros (varia de acordo com a seção), com o prazo de até duas semanas. Mas, você pode sempre renovar o seu empréstimo no próprio site da BU.

Horário de Funcionamento:


Segunda a sexta das 7:30 ás 22h

Sábados, Domingos e feriados: A sala de estudos individuais fica aberta das 8 às 17h

\subsection{LabUFSC}
No prédio da biblioteca, ao lado da entrada principal, fica o LabUFSC. Nele, há vários computadores com acesso a internet que podem ser usados gratuitamente pelos alunos. Para entrar, você só precisa apresentar sua carteirinha do RU na entrada.

Horário de Funcionamento:

Segunda a sexta das 8 às 22h

Sábados e Domingos das 8 às 17h

\subsection{PRAE} A Pró-Reitoria de Assuntos Estudantis – PRAE é um órgão voltado especialmente para programas e projetos ligados à política estudantil. É pra ela que se deve recorrer para saber sobre: Bolsas Estudantis, Auxílio Moradia, Moradia Estudantil, etc. Se você tem uma renda familiar bruta mensal de até 1,5 salário mínimo per capita (os que fizeram comprovação de renda já estão cadastrados), pode solicitar o seu Cadastro PRAE e então se inscrever para conseguir os benefícios. 

Os editais são lançados no site:

\url{http://prae.ufsc.br/editais-por-programa/}

\subsection{INE}
O Departamento de Informática e Estatística (INE) é o departamento responsável pelo nosso curso e o de Sistemas da Informação. Em seus cinco andares, possui cerca de 20 laboratórios (que serão apresentados) e é onde se encontram a secretaria e a coordenação do curso (térreo). Há também salas de professores e salas de aulas onde se realizam atividades práticas.


\section {Órgãos e Departamentos}

\subsection{RU}

Apesar das promessas de um novo RU e do crescente descaso e sucateamento do que já temos, continuamos com exatamente dois RU na UFSC de Floripa.

Um no campus Trindade, onde você possivelmente terá todas as aulas da sua graduação.
E um no campus do CCA (Centro de Ciências Agrárias), que fica num bairro um pouco mais distante chamado Itacorubi. A comida lá é muito mais temperada e frequentemente melhor que o do campus Trindade. Não deixe de provar quando tiver oportunidade.

O preço padrão do RU é R\$1,50. Pode ser gratuito em casos específicos como na comprovação de baixa renda (basta requisitar na PRAE).
O preço do RU também pode ser mais caro nos casos para professores, servidores e pessoas externas da universidade.
Sim, todos podem comer no RU, porém se a pessoa não tiver vínculo com a UFSC (ou tiver quebrado/esquecido o cartão do RU) precisa pagar o preço da refeição sem o subsídio da universidade(que ainda é muito mais barato que qualquer alternativa de refeição completa nos arredores).

Horário:
Segunda a sexta: 11h às 13:30h e das 17h às 19h

Sábados, domingos e feriados: 11h às 13:00h e das 17h às 19h

Cardápio:
O Cardápio do RU é disponibilizado semanalmente toda segunda-feira através do site \href{https://ru.ufsc.br/ru/}{ru.ufsc.br/ru/} e diariamente no instagram \href{https://www.instagram.com/ru360ufsc/}{@ru360ufsc}.

Também existe o amado \href{https://t.me/quibebot}{@quibebot} no Telegram, porém ele precisa de uma pequena manutenção a cada vez que a formatação do PDF do cardápio disponibilizado é modificado, geralmente a cada novo semestre. Aspirantes a heróis do curso, contatem o CALICO.

O cardápio é o mais provável, mas não garantido, não é incomum o cardápio mudar sem aviso prévio, mas normalmente, quando mudado, é para um prato similar ou Sassami.
A comida normalmente é composta por Arroz parboilizado e integral, feijão ou lentilha, um complemento variado (Farofa, batata palha, legume etc) uma porção de proteína, saladas e sobremesa. 
Dicas sobre o RU: 

Tome cuidado, o RU possui duas filas, uma na entrada frontal, que se desloca pela lateral do conviva, e outra que vai da porta dos fundos, pela lateral e saída do ru, até na frente da secretaria do RU, não misturem as duas, vai dar problema e chatice.
Se possível, priorize horários com menos movimento, como antes do 12:00 ou após 12:30.
Utilize todas as catracas disponíveis, não fique esperando alguém passar na catraca se tem uma disponível do lado.

\subsubsection{Carteirinha} Para conseguir acessar o RU, é necessário uma carteirinha e passes.
Na primeira semana costumava ser tolerado que os calouros apresentem seus atestados de matrícula para conseguirem entrar, já que as filas para fazer a carteirinha presencialmente eram muito grandes.
Atualmente a solicitação do cartão é apenas online e não é possível retirar o cartão no mesmo dia.
Então favor encher o saco da secretaria do RU para voltar o período de tolerância.

A solicitação é realizada no seguinte endereço: \href{https://ru.ufsc.br/solicitacao-online/}{ru.ufsc.br/solicitacao-online}.
Escolha uma foto boa, ela vai ficar na sua carteirinha para tudo, e se for recusada, vai atrasar o processo todo.

Todos os dados necessários para preencher a requisição podem ser encontrados no seu CAGR.

\subsubsection{Passes} 
Todos passes do cartão são recarregados apenas online: \href{https://ru.ufsc.br/pagamento-por-pix/}{ru.ufsc.br/ pagamento-por-pix}.
Enquanto que os passes físicos para pessoas externas são comprados na secretaria do RU (ao lado do RU) no caso do campus Trindade. No campus do CCA, esses passes são comprados na cantina.

\subsection{BU}
A Biblioteca Universitária (BU, lê-se “bê-u”) é o local de estudos mais usado pelos estudantes. Lá, você consegue total silêncio e tem a sua disposição um acervo enorme. É utilizada também para estudos em grupos. Se você quiser ainda mais concentração, a BU disponibiliza salas de estudos individuais, no piso térreo.

Para fazer um empréstimo, você deve ir até o balcão de atendimento com um documento de identificação oficial  e fazer seu cadastro. Com o cadastro feito, você pode consultar o acervo online, disponível no site bu.ufsc.br e localizar o seu livro. Depois, é só levar seu livro ao balcão e efetuar o empréstimo. Podem ser emprestados até cerca de 10 livros (varia de acordo com a seção), com o prazo de até duas semanas. Mas, você pode sempre renovar o seu empréstimo no próprio site da BU.

Horário de Funcionamento:


Segunda a sexta das 7:30 ás 22h

Sábados, Domingos e feriados: A sala de estudos individuais fica aberta das 8 às 17h

\subsection{LabUFSC}
No prédio da biblioteca, ao lado da entrada principal, fica o LabUFSC. Nele, há vários computadores com acesso a internet que podem ser usados gratuitamente pelos alunos. Para entrar, você só precisa apresentar sua carteirinha do RU na entrada.

Horário de Funcionamento:

Segunda a sexta das 8 às 22h

Sábados e Domingos das 8 às 17h

\subsection{PRAE} A Pró-Reitoria de Assuntos Estudantis – PRAE é um órgão voltado especialmente para programas e projetos ligados à política estudantil. É pra ela que se deve recorrer para saber sobre: Bolsas Estudantis, Auxílio Moradia, Moradia Estudantil, etc. Se você tem uma renda familiar bruta mensal de até 1,5 salário mínimo per capita (os que fizeram comprovação de renda já estão cadastrados), pode solicitar o seu Cadastro PRAE e então se inscrever para conseguir os benefícios. 

Os editais são lançados no site:

\url{http://prae.ufsc.br/editais-por-programa/}

\subsection{INE}
O Departamento de Informática e Estatística (INE) é o departamento responsável pelo nosso curso e o de Sistemas da Informação. Em seus cinco andares, possui cerca de 20 laboratórios (que serão apresentados) e é onde se encontram a secretaria e a coordenação do curso (térreo). Há também salas de professores e salas de aulas onde se realizam atividades práticas.



\newgeometry{margin=0pt}
\begin{figure}[p]
    \centering
    \includegraphics[width=\paperwidth, height=\paperheight]{mapa_UFSC_2014_1500x2121}
\end{figure}
\restoregeometry

\newpage
\thispagestyle{empty} % No header/footer on the blank page
\mbox{} % Adds an empty box, effectively creating a blank page
\newpage

\clearpage


% Contra-Capa
\thispagestyle{empty}
\begin{tikzpicture}[thick,scale=0.05,every node/.style={scale=0.26},overlay]
  \node[anchor=north east,inner sep=0pt] at ($(current page.south west)+(90cm,-277cm)$) {
     \includegraphics{CALICO}
  };
  
\end{tikzpicture}

\begin{tikzpicture}[thick,scale=0.05,every node/.style={scale=0.12},overlay]
    \node[anchor=north west,inner sep=0pt] at ($(current page.south west)+(130cm,-233cm)$) {
     \includegraphics{p2p}
  };
  
\end{tikzpicture}
% /Contra-Capa

\end{document}
