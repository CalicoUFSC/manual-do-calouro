\subsubsection{Ônibus}
Em Florianópolis, o único meio de transporte público disponível é o ônibus. As linhas são todas baseadas em terminais integrados espalhados pela cidade, sendo os mais próximos da UFSC o TITRI (Terminal de Integração da Trindade) e o TICEN (Terminal de Integração do Centro), desse jeito, se você pega um ônibus e desce em um terminal, pode entrar em outro ônibus e não pagar nada! Caso você possua qualquer um dos cartões da empresa, é possível integrar (lê-se, passar a catraca do ônibus sem pagar) em qualquer ponto da cidade, bastando utilizar o mesmo cartão dentro de um período estipulado (atualmente esse período é de aproximadamente 3h).

Além disso, os ônibus saem de seus respectivos lugares em horários específicos, e são, geralmente, bem pontuais.
Portanto, antes de sair de casa, verifique os horários do seu ônibus \href{consorciofenix.com.br}{consorciofenix .com.br}.
Ou baixe o app \href{https://play.google.com/store/search?q=floripa%20no%20ponto}{Floripa no Ponto 2.0}, esse app é péssimo para pesquisar qual ônibus pegar do ponto A ao ponto B, mas fornece a localização em tempo real de quase todos ônibus. Isso lhe poupará uma boa espera. Há também apps variados de consulta de horários disponíveis para serem baixados (o Google Maps é bem prático).

\subsubsection{Cartão de Estudante}

Você também pode fazer o seu cartão de estudante e pegar apenas meia tarifa.
Para isso, você precisará ir ao SETUF, localizado no TICEN, levando os documentos necessários que estão listados em \href{https://www.setuf.com.br/facil/perguntas-frequentes/perguntas-estudante/}{https://www.setuf.com.br/facil/perguntas-frequentes/perguntas-estudante/}.

O comprovante de matrícula disponível no seu \href{https://cagr.sistemas.ufsc.br/relatorios/aluno/atestadoMatricula?download}{CAGR},
que é assinado digitalmente como é possível de se verificar no rodapé, constando como "regularmente matriculado" deveria ser suficiente para comprovar seu vínculo acadêmico.
Porém é comum haver atrito para ser aceito, nesses casos procure auxílio do movimento estudantil (CALICO e DCE) ou da coordenação do curso (cco@contato. ufsc.br).

É altamente recomendável que se faça o cartão.
Para recadastros e recargas do cartão, é possível fazer tanto em qualquer um dos terminais integrados quanto inteiramente no aplicativo \href{https://play.google.com/store/apps/details?id=br.com.henkoti.empresa1.smpay}{SI.GO}
