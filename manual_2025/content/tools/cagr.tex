\subsection{CAGR e Moodle}

O Sistema de Controle Acadêmico da Graduação, CAGR, é a principal ferramenta para acompanhar seu desempenho no curso. Lá, você tem acesso ao fórum da graduação, grade de horários, atestado de matrícula, entre outras coisas (os mais curiosos vão ter vastos dados da universidade para encontrar). Para se cadastrar, entre em cagr.ufsc.br e clique em “primeiro acesso”.

O Moodle contém as turmas em que você está matriculado (podendo alguma não estar, se o tópico não foi aberto), sendo possível então o professor de cada disciplina disponibilizar slides, fazer um controle de notas ou faltas, ou até mesmo aplicar provas por essa plataforma. É onde, geralmente, seu professor coloca o seu material de estudo e passa recados importantes. Como o acesso é unificado, basta entrar em moodle.ufsc.br e entrar com o que você previamente se cadastrou no CAGR. Nada como um sentimento de “notas da prova no moodle” (desespero).
