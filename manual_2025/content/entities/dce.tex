\subsection{DCE}
Além dos Centros Acadêmicos, existe o Diretório Central dos Estudantes, que engloba os estudantes de todos os campi da UFSC.

O DCE tem um carater muito mais abrangente e impactante.
Por exemplo, é por ele que disputamos os espaços que todos utilizamos, como o RU e a BU.
Nos informamos sobre o que ocorre nos outros cantos da universidade.
E decidimos as lutas que travaremos.

Seguindo a lógica de reuniões acima de cargos, existem duas reuniões deliberativas importantes no DCE.

Uma é o Conselho de Entidades de Base (CEB), onde todos os estudantes são convocados periodicamente para debater, mas só se tem direito ao voto um representante por centro acadêmico.

E a normalmente entendida como instância máxima e convocada apenas para pautas de máxima importância, a Assembleia Geral Estudantil do DCE.
Nessa instância todos os estudantes presentes tem direito a um voto.

Para finalizar a exposição nesse manual sobre reuniões deliberativas importantes, a maior instância institucional da UFSC inteira é o Conselho Universtário (CUn).
Presidido pelo reitor, é aqui onde são decididos os regimentos e ofícios mais importantes da UFSC,
como questões relacionadas a matrícula e permanência dos estudantes por exemplo.
Apesar de ser aberto para todo o público da UFSC participar, apenas os conselheiros tem direito a voto.
Através do nosso querido DCE conquistamos cadeiras para sermos representados, disputando votos diretamente com o reitor e diretores de cada centro.
