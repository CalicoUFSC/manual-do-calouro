\subsection{CALICO}
Cada curso possui um Centro Acadêmico, o nosso se chama CALICO - Centro Acadêmico Livre de Ciência da Computação. O CALICO assim como vários outros Centros Acadêmicos na UFSC se tornaram livres no período da ditadura empresarial-militar, onde não aceitamos sermos geridos pelos indicados dos militares. Então ainda que sejamos uma entidade oficial da universidade, temos total autonomia para nos articularmos entre nós.

Para entender melhor nossos princípios, composição e organização basta consultar nosso \href{https://github.com/calicoufsc/estatuto}{estatuto}, que pode ser encontrado em: \href{https://github.com/calicoufsc/estatuto}{https://github.com/calicoufsc/estatuto}

Mas afinal o que é feito com essa entidade? Ora, tudo aquilo que o estudante do curso precisar.

\subsubsection{Integração}

O primeiro contato com a entidade costuma ser algum evento de integração, principalmente na recepção dos calouros que esse manual faz parte. Há a tradição de organizar os seguintes eventos (não necessariamente todos em todo semestre):
\begin{itemize}
    \item Noite de jogos: Noite repleta de jogos de tabuleiro, os melhores são os de deduções sociais como Secret Hitler ou Lobisomem.
    \item Chicco: Churrasco de Integração dos Calouros de Computação, onde o pão com linguiça reina soberano.
    \item Jose: Jogos Sedentários de Computação, esse é composto de jogos competitivos (beer pong, truco, magic, lol, cs e muito mais)
    \item Apadrinhamento de calouros: Os veteranos mais prestativos se disponibilizam a ajudar os calouros de maneira mais próxima e particular.
    \item Emplacamento: Momento que os calouros falam sobre si e ganham uma nova identidade própria no curso cunhado num novo nome (é muito difícil de lembrar quem é Lucas ou Arthur, mas todos vão lembrar do seu novo nome se você contar uma história única e marcante o suficiente sobre si). 
    \item Além de muitos outros que não ganharam um nome próprio (karaokê, trilha, escape room...)
\end{itemize}

\subsubsection{Formação}

Não só de entretenimento vive o estudante. As aulas ficam muito mais agradáveis quando não temos que aprender o conteúdo das disciplinas junto de uma linguagem de programação, sistema operacional ou ferramenta de versionamento (aquilo que te liberta do trabalho-final123-finalmesmo.py).

Sim, a linguagem mais "oficial" do curso é Python. Mais uma conquista do CALICO, antes sofriamos com Java (quem acha Java melhor que Python tá errado). Qual deveria ser a próxima linguagem pela qual deveríamos nos empenhar?

Falando em empenhos, os eventos de formação mais característicos do curso são:
\begin{itemize}
    \item Minicurso de Linux: Quanto antes você se familiarizar com esse sistema lindo melhor, mas ele se torna inevitável de usar na terceira fase.
    \item Cursos de programação: Nas próximas fases surgem diversas novas linguagens e os professores não tem muito tempo hábil para ensina-las em sala de aula.
    \item Oficinas: Seja de uma biblioteca, conceito ou ferramenta (melhor lugar pra aprender git).
    \item Rodas de conversa: Sempre empolgante debater com especialistas de outras áreas.
    \item Mutirão de instalação de Linux: Máquina virtual não é vida, abrace um dos incontáveis sabores de Linux (Arch é o melhor, mas é um dementador do seu tempo no início).
\end{itemize}

Também existe um mega evento que é organizado em conjunto com outras entidades, a SECCOM - Semana Acadêmica de Computação e Sistemas. Temos uma subseção dedicada à ela mais adiante no manual.

\subsubsection{Representação}

A UFSC possui várias instâncias políticas. E, assim como no CALICO, a maioria delas tem como autoridade máxima de deliberação não um único cargo (como o diretor ou reitor), mas um conjunto de pessoas que se reúnem periodicamente para votar as decisões. No caso do CALICO, nada está acima da assembleia devidamente descrita no estatuto (nem mesmo a diretoria), onde cada estudante de computação da UFSC presente tem direito a um voto.

Já no caso das instâncias institucionais, não é possível que todos estejam presentes, então são enviados representantes. O CALICO tem o poder de apontar membros para os seguintes órgãos deliberativos:
\begin{itemize}
    % \item Colegiado do Curso: Esse é o orgão que nos diz respeito de maneira mais direta, dentre os casos decididos aqui estão as quebras de pré-requisitos para disciplinas, jubilamentos e alterações no currículo.
    \item Colegiado do Curso: Esse é o orgão que nos diz respeito de maneira mais direta, dentre os casos decididos aqui estão as quebras de requisitos para disciplinas, jubilamentos e alterações no currículo.
    \item Colegiado do Departamento: Os assuntos daqui já são mais comumentes indiferentes a nós, geralmente tratam mais sobre licenças e outras questões burocráticas dos professores, mas também é onde são decididos questões relacionadas ao prédio do INE. 
    \item Conselho do CTC: Nós, CALICO, indicamos um representante através do CETEC (outra entidade estudantil). Nessas reuniões são decididos assuntos maiores como os requisitos para se exercer um cargo do centro ou quais estratégias adotar para licitar nossa negligênciada (mas jamais esquecida) cantina. O debate é tão cerceado que nem mesmo professores de fora do conselho costumam estar a par do que acontece.
\end{itemize}


A estrutura da UFSC pode ser mais assustadora do que realmente é num primeiro momento. Mas o CALICO é esse espaço para sermos acolhidos, instigados e desenvolver o que julgarmos de mais importante para o estudante de computação.

A sede da nossa entidade se localiza subindo as escadas ao lado do Bar do CTC, no Centro de Convivência do CTC (também chamado de CETEC, apesar desse ser o nome da entidade que ocupa o andar de cima do prédio), onde são mantidas todas as salas de centros acadêmicos do CTC. A presença de todos é mais que bem-vinda para estudar, conversar ou jogar PS4 :)

\subsubsection{Canais e mídias sociais}
Para ficar ligado e saber quando vão ocorrer os eventos ou pra ficar por dentro do que tá rolando na computação e na UFSC, siga nosso instagram \href{https://www.instagram.com/calicoufsc/}{@calicoufsc}, entre no nosso sevidor do \href{https://discord.gg/YPU4MHmp}{Discord}, se inscreva nos nossos canais de notícias no Telegram (\href{https://t.me/noticiascalico}{@noticiascalico}) ou comunidade no Whatsapp (\href{https://chat.whatsapp.com/KPH3O3ZBJEd9Nwijvbwt9d}{CALICO})!
