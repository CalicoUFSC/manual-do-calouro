\subsection{Empresas Juniores}

Empresas juniores existem devido ao cerceamento do emprego formal.

Já se perguntou como as empresas possuem demanda para nossa mão de obra especializada ainda que sem experiência, mas não estão dispostas a nos remunerar pelo nosso trabalho?

Talvez falta de dinheiro para nos contratar?
Certamente os mantenedores (\href{https://brasiljunior.org.br/}{brasiljunior.org.br}) gigantescos Ambev e Bradesco estão de mãos atadas. 

Ou então por que os estudantes não decidem trabalhar para si próprios ou para grupos sociais até adquirirem experiência suficiente para serem consideráveis "rentáveis"\ o suficiente para que as empresas os remunerem? 

Falta de domínio de alguma ferramenta pode ser adquirido através oficinas, desenvolvimento de projetos pessoais ou contribuindo em uma comunidade de software livre.

Melhora-se a capacidade de colaboração em equipe se envolvendo com movimentos sociais ou organizações de eventos (como Amnésia e SECCOM).

Também se melhora a comunicação e clareza de ideias expostas ao praticar monitoria, ministrar minicursos, apresentar palestras ou, principalmente, participar de reuniões deliberativas que se tenha voz ativa.

A questão é que a viabilidade de vários projetos (principalmente aqueles maiores) necessitam de um investimento que um estudante ou grupo social não costuma ter a sua disposição. Esse tipo de financiamento só é possível a partir do estado ou de empresas.

Aqueles que não conseguem adquirir uma bolsa de extensão, encontrar uma vaga de estágio e precisam iniciar uma carreira logo são cada vez mais pressionados a se submeter ao mercado de trabalho como voluntários.

É nesse contexto que as empresas juniores são possíveis.
São organizações "sem fins lucrativos"\ e com isenções fiscais, onde a maioria dos funcionários são estudantes universitários, normalmente sob supervisão de um professor.
Desenvolvem projetos e serviços para outras empresas e todo o dinheiro recebido deve ser reinvestido dentro da própria empresa júnior agregando ainda mais valor aos seus produtos.

\subsection{Pixel}

A empresa júnior (EJ) de Ciência da Computação juntamente com Sistemas de Informação é a Pixel.
Trabalha com o desenvolvimento de sites e e-commerce.

Conheça mais sobre a Pixel em \href{pixel.ufsc.br}{pixel.ufsc.br} e demais empresas juniores da UFSC em: \href{https://empresasjuniores.paginas.ufsc.br/lista-de-empresas-juniores-da-ufsc/}{empresasjuniores.paginas.ufsc.br/lista-de-empresas-juniores-da-ufsc}.
